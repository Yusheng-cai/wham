%% Generated by Sphinx.
\def\sphinxdocclass{report}
\documentclass[letterpaper,10pt,english]{sphinxmanual}
\ifdefined\pdfpxdimen
   \let\sphinxpxdimen\pdfpxdimen\else\newdimen\sphinxpxdimen
\fi \sphinxpxdimen=.75bp\relax

\PassOptionsToPackage{warn}{textcomp}
\usepackage[utf8]{inputenc}
\ifdefined\DeclareUnicodeCharacter
% support both utf8 and utf8x syntaxes
  \ifdefined\DeclareUnicodeCharacterAsOptional
    \def\sphinxDUC#1{\DeclareUnicodeCharacter{"#1}}
  \else
    \let\sphinxDUC\DeclareUnicodeCharacter
  \fi
  \sphinxDUC{00A0}{\nobreakspace}
  \sphinxDUC{2500}{\sphinxunichar{2500}}
  \sphinxDUC{2502}{\sphinxunichar{2502}}
  \sphinxDUC{2514}{\sphinxunichar{2514}}
  \sphinxDUC{251C}{\sphinxunichar{251C}}
  \sphinxDUC{2572}{\textbackslash}
\fi
\usepackage{cmap}
\usepackage[T1]{fontenc}
\usepackage{amsmath,amssymb,amstext}
\usepackage{babel}



\usepackage{times}
\expandafter\ifx\csname T@LGR\endcsname\relax
\else
% LGR was declared as font encoding
  \substitutefont{LGR}{\rmdefault}{cmr}
  \substitutefont{LGR}{\sfdefault}{cmss}
  \substitutefont{LGR}{\ttdefault}{cmtt}
\fi
\expandafter\ifx\csname T@X2\endcsname\relax
  \expandafter\ifx\csname T@T2A\endcsname\relax
  \else
  % T2A was declared as font encoding
    \substitutefont{T2A}{\rmdefault}{cmr}
    \substitutefont{T2A}{\sfdefault}{cmss}
    \substitutefont{T2A}{\ttdefault}{cmtt}
  \fi
\else
% X2 was declared as font encoding
  \substitutefont{X2}{\rmdefault}{cmr}
  \substitutefont{X2}{\sfdefault}{cmss}
  \substitutefont{X2}{\ttdefault}{cmtt}
\fi


\usepackage[Bjarne]{fncychap}
\usepackage{sphinx}

\fvset{fontsize=\small}
\usepackage{geometry}


% Include hyperref last.
\usepackage{hyperref}
% Fix anchor placement for figures with captions.
\usepackage{hypcap}% it must be loaded after hyperref.
% Set up styles of URL: it should be placed after hyperref.
\urlstyle{same}
\addto\captionsenglish{\renewcommand{\contentsname}{Contents:}}

\usepackage{sphinxmessages}
\setcounter{tocdepth}{1}



\title{Wham}
\date{Jan 21, 2021}
\release{0.0.1}
\author{Yusheng Cai}
\newcommand{\sphinxlogo}{\vbox{}}
\renewcommand{\releasename}{Release}
\makeindex
\begin{document}

\pagestyle{empty}
\sphinxmaketitle
\pagestyle{plain}
\sphinxtableofcontents
\pagestyle{normal}
\phantomsection\label{\detokenize{index::doc}}



\chapter{wham}
\label{\detokenize{wham:wham}}\label{\detokenize{wham::doc}}\label{\detokenize{wham::doc}}

\section{wham.lib}
\label{\detokenize{wham.lib:wham-lib}}\label{\detokenize{wham.lib::doc}}

\subsection{wham.lib.numeric}
\label{\detokenize{wham.lib:module-wham.lib.numeric}}\label{\detokenize{wham.lib:wham-lib-numeric}}\index{wham.lib.numeric (module)@\spxentry{wham.lib.numeric}\spxextra{module}}\index{safe\_divide() (in module wham.lib.numeric)@\spxentry{safe\_divide()}\spxextra{in module wham.lib.numeric}}

\begin{fulllineitems}
\phantomsection\label{\detokenize{wham.lib:wham.lib.numeric.safe_divide}}\pysiglinewithargsret{\sphinxcode{\sphinxupquote{wham.lib.numeric.}}\sphinxbfcode{\sphinxupquote{safe\_divide}}}{\emph{a}, \emph{b}}{}
Performs divide dividing a 0 in the denominator would be treated as 0
\begin{quote}\begin{description}
\item[{Parameters}] \leavevmode\begin{itemize}
\item {} 
\sphinxstyleliteralstrong{\sphinxupquote{a}} (\sphinxstyleliteralemphasis{\sphinxupquote{np.ndarray}}) \textendash{} a vector

\item {} 
\sphinxstyleliteralstrong{\sphinxupquote{b}} (\sphinxstyleliteralemphasis{\sphinxupquote{np.ndarray}}) \textendash{} a vector

\end{itemize}

\item[{Returns}] \leavevmode
A vector which is equal to a/b that ignores 0

\end{description}\end{quote}

\end{fulllineitems}

\index{safe\_log() (in module wham.lib.numeric)@\spxentry{safe\_log()}\spxextra{in module wham.lib.numeric}}

\begin{fulllineitems}
\phantomsection\label{\detokenize{wham.lib:wham.lib.numeric.safe_log}}\pysiglinewithargsret{\sphinxcode{\sphinxupquote{wham.lib.numeric.}}\sphinxbfcode{\sphinxupquote{safe\_log}}}{\emph{a}}{}
Safely performs logarithms

\end{fulllineitems}

\index{autograd\_logsumexp() (in module wham.lib.numeric)@\spxentry{autograd\_logsumexp()}\spxextra{in module wham.lib.numeric}}

\begin{fulllineitems}
\phantomsection\label{\detokenize{wham.lib:wham.lib.numeric.autograd_logsumexp}}\pysiglinewithargsret{\sphinxcode{\sphinxupquote{wham.lib.numeric.}}\sphinxbfcode{\sphinxupquote{autograd\_logsumexp}}}{\emph{a}, \emph{b=1}, \emph{axis=0}}{}
Performs logsumexp using the numpy from autograd
np.log(np.sum(a*np.exp(b)))
\begin{quote}\begin{description}
\item[{Parameters}] \leavevmode\begin{itemize}
\item {} 
\sphinxstyleliteralstrong{\sphinxupquote{a}} (\sphinxstyleliteralemphasis{\sphinxupquote{np.ndarray}}) \textendash{} The matrix/vector to be exponentiated

\item {} 
\sphinxstyleliteralstrong{\sphinxupquote{b}} (\sphinxstyleliteralemphasis{\sphinxupquote{np.ndarray}}) \textendash{} The number at which to multiply exp(a)

\item {} 
\sphinxstyleliteralstrong{\sphinxupquote{axis}} (\sphinxstyleliteralemphasis{\sphinxupquote{int}}) \textendash{} the axis at which to sum over

\end{itemize}

\item[{Returns}] \leavevmode
a matrix that is the logsumexp result of a \& b

\end{description}\end{quote}

\end{fulllineitems}



\subsection{wham.lib.wham\_utils}
\label{\detokenize{wham.lib:module-wham.lib.wham_utils}}\label{\detokenize{wham.lib:wham-lib-wham-utils}}\index{wham.lib.wham\_utils (module)@\spxentry{wham.lib.wham\_utils}\spxextra{module}}\index{read\_dat() (in module wham.lib.wham\_utils)@\spxentry{read\_dat()}\spxextra{in module wham.lib.wham\_utils}}

\begin{fulllineitems}
\phantomsection\label{\detokenize{wham.lib:wham.lib.wham_utils.read_dat}}\pysiglinewithargsret{\sphinxcode{\sphinxupquote{wham.lib.wham\_utils.}}\sphinxbfcode{\sphinxupquote{read\_dat}}}{\emph{file\_path}}{}
Function that reads the .dat from INDUS simulations
\begin{quote}\begin{description}
\item[{Parameters}] \leavevmode
\sphinxstyleliteralstrong{\sphinxupquote{file\_path}} (\sphinxstyleliteralemphasis{\sphinxupquote{str}}) \textendash{} the path to the file

\item[{Returns}] \leavevmode
an numpy array that contains the N and Ntilde from the INDUS simulation (N, Ntilde) \sphinxhyphen{}\textgreater{} where both are of shape (nobs,)

\end{description}\end{quote}

\end{fulllineitems}

\index{make\_bins() (in module wham.lib.wham\_utils)@\spxentry{make\_bins()}\spxextra{in module wham.lib.wham\_utils}}

\begin{fulllineitems}
\phantomsection\label{\detokenize{wham.lib:wham.lib.wham_utils.make_bins}}\pysiglinewithargsret{\sphinxcode{\sphinxupquote{wham.lib.wham\_utils.}}\sphinxbfcode{\sphinxupquote{make\_bins}}}{\emph{data}, \emph{min}, \emph{max}, \emph{bins=101}}{}
A function that bins some data within min and max
\begin{quote}\begin{description}
\item[{Parameters}] \leavevmode\begin{itemize}
\item {} 
\sphinxstyleliteralstrong{\sphinxupquote{data}} (\sphinxstyleliteralemphasis{\sphinxupquote{np.ndarray}}) \textendash{} the data that you want to bin, pass in numpy array (shape(N,))

\item {} 
\sphinxstyleliteralstrong{\sphinxupquote{min}} (\sphinxstyleliteralemphasis{\sphinxupquote{float}}) \textendash{} minimum of the bins

\item {} 
\sphinxstyleliteralstrong{\sphinxupquote{max}} (\sphinxstyleliteralemphasis{\sphinxupquote{float}}) \textendash{} maximum of the bins

\item {} 
\sphinxstyleliteralstrong{\sphinxupquote{bins}} (\sphinxstyleliteralemphasis{\sphinxupquote{int}}) \textendash{} number of bins to make

\end{itemize}

\item[{Returns}] \leavevmode
tuple of (bins,binned\_vec)

\end{description}\end{quote}

\end{fulllineitems}

\index{cov\_fi() (in module wham.lib.wham\_utils)@\spxentry{cov\_fi()}\spxextra{in module wham.lib.wham\_utils}}

\begin{fulllineitems}
\phantomsection\label{\detokenize{wham.lib:wham.lib.wham_utils.cov_fi}}\pysiglinewithargsret{\sphinxcode{\sphinxupquote{wham.lib.wham\_utils.}}\sphinxbfcode{\sphinxupquote{cov\_fi}}}{\emph{wji}, \emph{Ni}}{}
This is a function that calculates the covariance matrix of fi where
fi = \sphinxhyphen{}ln(Qi/Q0).
\begin{quote}\begin{description}
\item[{Parameters}] \leavevmode\begin{itemize}
\item {} 
\sphinxstyleliteralstrong{\sphinxupquote{wji}} (\sphinxstyleliteralemphasis{\sphinxupquote{np.ndarray}}) \textendash{} the weight matrices of different simulations (N,k) where N=number of observations

\item {} 
\sphinxstyleliteralstrong{\sphinxupquote{total}}\sphinxstyleliteralstrong{\sphinxupquote{, }}\sphinxstyleliteralstrong{\sphinxupquote{k=number of simulations}} (\sphinxstyleliteralemphasis{\sphinxupquote{in}}) \textendash{} 

\item {} 
\sphinxstyleliteralstrong{\sphinxupquote{Ni}} (\sphinxstyleliteralemphasis{\sphinxupquote{np.ndarray}}) \textendash{} the number of observations in each simulation

\end{itemize}

\end{description}\end{quote}
\begin{description}
\item[{returns}] \leavevmode
covariance of fi in shape (k,k)

\end{description}

\end{fulllineitems}



\section{wham.Bwham}
\label{\detokenize{wham:module-wham.Bwham}}\label{\detokenize{wham:wham-bwham}}\index{wham.Bwham (module)@\spxentry{wham.Bwham}\spxextra{module}}\index{Bwham (class in wham.Bwham)@\spxentry{Bwham}\spxextra{class in wham.Bwham}}

\begin{fulllineitems}
\phantomsection\label{\detokenize{wham:wham.Bwham.Bwham}}\pysiglinewithargsret{\sphinxbfcode{\sphinxupquote{class }}\sphinxcode{\sphinxupquote{wham.Bwham.}}\sphinxbfcode{\sphinxupquote{Bwham}}}{\emph{xji}, \emph{Ntwiddle}, \emph{Ni}, \emph{k}, \emph{min\_}, \emph{max\_}, \emph{bins=101}, \emph{beta=0.4036}}{}
Bases: \sphinxcode{\sphinxupquote{object}}
\begin{quote}\begin{description}
\item[{Parameters}] \leavevmode\begin{itemize}
\item {} 
\sphinxstyleliteralstrong{\sphinxupquote{xji}} (\sphinxstyleliteralemphasis{\sphinxupquote{np.ndarray}}) \textendash{} all the observation in dataset (Ntot,)

\item {} 
\sphinxstyleliteralstrong{\sphinxupquote{Ntwiddle}} (\sphinxstyleliteralemphasis{\sphinxupquote{np.ndarray}}) \textendash{} The Ntwiddle for all the biased simulations (S\sphinxhyphen{}1,)

\item {} 
\sphinxstyleliteralstrong{\sphinxupquote{Ni}} (\sphinxstyleliteralemphasis{\sphinxupquote{np.ndarray}}) \textendash{} The number of observations in each simulation (S,)

\item {} 
\sphinxstyleliteralstrong{\sphinxupquote{k}} (\sphinxstyleliteralemphasis{\sphinxupquote{float}}) \textendash{} The parameter for the harmonic potential where U=0.5*k(x\sphinxhyphen{}xstar)**2

\item {} 
\sphinxstyleliteralstrong{\sphinxupquote{min\_}} (\sphinxstyleliteralemphasis{\sphinxupquote{float}}) \textendash{} the minimum of the bins

\item {} 
\sphinxstyleliteralstrong{\sphinxupquote{max\_}} (\sphinxstyleliteralemphasis{\sphinxupquote{float}}) \textendash{} the maximum of the bins

\item {} 
\sphinxstyleliteralstrong{\sphinxupquote{bins}} (\sphinxstyleliteralemphasis{\sphinxupquote{int}}) \textendash{} number of bins

\item {} 
\sphinxstyleliteralstrong{\sphinxupquote{beta}} (\sphinxstyleliteralemphasis{\sphinxupquote{float}}) \textendash{} beta=1/kbT where the default value is at T=298K

\item {} 
\sphinxstyleliteralstrong{\sphinxupquote{Ml}} (\sphinxstyleliteralemphasis{\sphinxupquote{np.ndarray}}) \textendash{} Number of simulations in the lth bin from all simulations (M,)

\item {} 
\sphinxstyleliteralstrong{\sphinxupquote{Wil}} (\sphinxstyleliteralemphasis{\sphinxupquote{np.ndarray}}) \textendash{} Biased energy from different simulations, 0 for unbiased simulation

\item {} 
\sphinxstyleliteralstrong{\sphinxupquote{fi0}} (\sphinxstyleliteralemphasis{\sphinxupquote{np.ndarray}}) \textendash{} An array of zero’s which can be used as the initial guess for the optimization

\end{itemize}

\end{description}\end{quote}
\index{initialize() (wham.Bwham.Bwham method)@\spxentry{initialize()}\spxextra{wham.Bwham.Bwham method}}

\begin{fulllineitems}
\phantomsection\label{\detokenize{wham:wham.Bwham.Bwham.initialize}}\pysiglinewithargsret{\sphinxbfcode{\sphinxupquote{initialize}}}{}{}
Function that initializes some variables
\begin{quote}\begin{description}
\item[{Returns}] \leavevmode
\begin{enumerate}
\sphinxsetlistlabels{\arabic}{enumi}{enumii}{}{.}%
\item {} 
Ml= observations in the lth bin from all simulations (M,)

\item {} 
Wil= The energy matrix where Wil = 0.5*beta*ki*(xl\sphinxhyphen{}xi)**2

\item {} 
bins= the binned vector (M,)

\end{enumerate}


\end{description}\end{quote}

\end{fulllineitems}

\index{self\_consistent() (wham.Bwham.Bwham method)@\spxentry{self\_consistent()}\spxextra{wham.Bwham.Bwham method}}

\begin{fulllineitems}
\phantomsection\label{\detokenize{wham:wham.Bwham.Bwham.self_consistent}}\pysiglinewithargsret{\sphinxbfcode{\sphinxupquote{self\_consistent}}}{\emph{tol=1e\sphinxhyphen{}10}, \emph{maxiter=100000.0}, \emph{print\_every=\sphinxhyphen{}1}}{}
Implementation of self consistent solver of binned wham
\begin{quote}\begin{description}
\item[{Parameters}] \leavevmode\begin{itemize}
\item {} 
\sphinxstyleliteralstrong{\sphinxupquote{tol}} (\sphinxstyleliteralemphasis{\sphinxupquote{float}}) \textendash{} the tolerance for convergence  (default 1e\sphinxhyphen{}10)

\item {} 
\sphinxstyleliteralstrong{\sphinxupquote{maxiter}} (\sphinxstyleliteralemphasis{\sphinxupquote{int}}) \textendash{} maximum iterations possible for self consistent solver (default 1e5)

\item {} 
\sphinxstyleliteralstrong{\sphinxupquote{print\_every}} (\sphinxstyleliteralemphasis{\sphinxupquote{int}}) \textendash{} how many iterations to print result. A number less than 0 indicates never. (default \sphinxhyphen{}1)

\end{itemize}

\item[{Returns}] \leavevmode
\begin{enumerate}
\sphinxsetlistlabels{\arabic}{enumi}{enumii}{}{.}%
\item {} 
fi = \sphinxhyphen{}log(Zi/Z0)

\item {} 
\sphinxhyphen{}log(pl) = Free energy distribution in each bin

\item {} 
pl = probability as each bin l

\end{enumerate}


\end{description}\end{quote}

\end{fulllineitems}

\index{Maximum\_likelihood() (wham.Bwham.Bwham method)@\spxentry{Maximum\_likelihood()}\spxextra{wham.Bwham.Bwham method}}

\begin{fulllineitems}
\phantomsection\label{\detokenize{wham:wham.Bwham.Bwham.Maximum_likelihood}}\pysiglinewithargsret{\sphinxbfcode{\sphinxupquote{Maximum\_likelihood}}}{\emph{ftol=2.22e\sphinxhyphen{}09}, \emph{gtol=1e\sphinxhyphen{}05}, \emph{maxiter=15000}, \emph{maxfun=15000}, \emph{iprint=\sphinxhyphen{}1}}{}~\begin{quote}\begin{description}
\item[{Parameters}] \leavevmode\begin{itemize}
\item {} 
\sphinxstyleliteralstrong{\sphinxupquote{ftol}} (\sphinxstyleliteralemphasis{\sphinxupquote{float}}) \textendash{} tolerance parameter as set forth by scipy.minimize’s ‘L\sphinxhyphen{}BFGS\sphinxhyphen{}B’ option (default 2.22e\sphinxhyphen{}09)

\item {} 
\sphinxstyleliteralstrong{\sphinxupquote{gtol}} (\sphinxstyleliteralemphasis{\sphinxupquote{float}}) \textendash{} tolerance parameter as set forth by scipy.minimize’s ‘L\sphinxhyphen{}BFGS\sphinxhyphen{}B’ option (default 1e\sphinxhyphen{}05)

\item {} 
\sphinxstyleliteralstrong{\sphinxupquote{maxiter}} (\sphinxstyleliteralemphasis{\sphinxupquote{int}}) \textendash{} Maximum iteration as set forth by scipy.minimize (default 15000)

\item {} 
\sphinxstyleliteralstrong{\sphinxupquote{maxfun}} (\sphinxstyleliteralemphasis{\sphinxupquote{int}}) \textendash{} Maximum function evaluation as set forth by scipy.minimize (default 15000)

\item {} 
\sphinxstyleliteralstrong{\sphinxupquote{iprint}} (\sphinxstyleliteralemphasis{\sphinxupquote{int}}) \textendash{} The interval between which the user wants result to be printed. A number less than 0

\item {} 
\sphinxstyleliteralstrong{\sphinxupquote{never.}} (\sphinxstyleliteralemphasis{\sphinxupquote{indicates}}) \textendash{} 

\end{itemize}

\item[{Returns}] \leavevmode
\begin{enumerate}
\sphinxsetlistlabels{\arabic}{enumi}{enumii}{}{.}%
\item {} 
fi= \sphinxhyphen{}ln(Zi/Z0) (S,)

\item {} 
Fl= \sphinxhyphen{}log(pl) free energy at the lth bin (M,)

\item {} 
pl=probability at the lth bin (M,)

\end{enumerate}


\end{description}\end{quote}

\end{fulllineitems}

\index{get\_pil() (wham.Bwham.Bwham method)@\spxentry{get\_pil()}\spxextra{wham.Bwham.Bwham method}}

\begin{fulllineitems}
\phantomsection\label{\detokenize{wham:wham.Bwham.Bwham.get_pil}}\pysiglinewithargsret{\sphinxbfcode{\sphinxupquote{get\_pil}}}{}{}
Function that obtains the matrix of pil which is defined as
\begin{quote}\begin{description}
\item[{Returns}] \leavevmode
matrix with shape (S,M)

\item[{Return type}] \leavevmode
pil(np.ndarray)

\end{description}\end{quote}

\end{fulllineitems}


\end{fulllineitems}

\index{Bwham\_NLL\_eq() (in module wham.Bwham)@\spxentry{Bwham\_NLL\_eq()}\spxextra{in module wham.Bwham}}

\begin{fulllineitems}
\phantomsection\label{\detokenize{wham:wham.Bwham.Bwham_NLL_eq}}\pysiglinewithargsret{\sphinxcode{\sphinxupquote{wham.Bwham.}}\sphinxbfcode{\sphinxupquote{Bwham\_NLL\_eq}}}{\emph{x}, \emph{Ni}, \emph{Ml}, \emph{Wil}}{}~\begin{quote}\begin{description}
\item[{Parameters}] \leavevmode\begin{itemize}
\item {} 
\sphinxstyleliteralstrong{\sphinxupquote{x}} \textendash{} shape (S,)

\item {} 
\sphinxstyleliteralstrong{\sphinxupquote{Ni}} \textendash{} Number of data counts in simulation i (S,)

\item {} 
\sphinxstyleliteralstrong{\sphinxupquote{Ml}} \textendash{} Number of data in from simulation i=1,…,S in bin l (M,)

\item {} 
\sphinxstyleliteralstrong{\sphinxupquote{Wil}} \textendash{} 0.5*k*beta*(n\sphinxhyphen{}nstar)**2 (S,M)

\end{itemize}

\item[{Returns}] \leavevmode
the value of the negative likelihood function

\end{description}\end{quote}

\end{fulllineitems}



\section{wham.Uwham}
\label{\detokenize{wham:module-wham.Uwham}}\label{\detokenize{wham:wham-uwham}}\index{wham.Uwham (module)@\spxentry{wham.Uwham}\spxextra{module}}\index{Uwham (class in wham.Uwham)@\spxentry{Uwham}\spxextra{class in wham.Uwham}}

\begin{fulllineitems}
\phantomsection\label{\detokenize{wham:wham.Uwham.Uwham}}\pysiglinewithargsret{\sphinxbfcode{\sphinxupquote{class }}\sphinxcode{\sphinxupquote{wham.Uwham.}}\sphinxbfcode{\sphinxupquote{Uwham}}}{\emph{xji}, \emph{k}, \emph{Ntwiddle}, \emph{Ni}, \emph{beta=0.4036}}{}
Bases: \sphinxcode{\sphinxupquote{object}}

A class that performs the unbinned calculations
\begin{quote}\begin{description}
\item[{Parameters}] \leavevmode\begin{itemize}
\item {} 
\sphinxstyleliteralstrong{\sphinxupquote{xji}} (\sphinxstyleliteralemphasis{\sphinxupquote{np.ndarray}}) \textendash{} all the observations inclduing biased and unbiased simulations (Ntot,)

\item {} 
\sphinxstyleliteralstrong{\sphinxupquote{k}} (\sphinxstyleliteralemphasis{\sphinxupquote{float}}) \textendash{} the parameter in harmonic potential where U(x)=0.5*k*(x\sphinxhyphen{}xstar)**2

\item {} 
\sphinxstyleliteralstrong{\sphinxupquote{Ntwiddle}} (\sphinxstyleliteralemphasis{\sphinxupquote{np.ndarray}}) \textendash{} The Ntwiddle of all the biased simulations (S\sphinxhyphen{}1,)

\item {} 
\sphinxstyleliteralstrong{\sphinxupquote{Ni}} (\sphinxstyleliteralemphasis{\sphinxupquote{np.ndarray}}) \textendash{} Number of observations in each simulations (S,)

\item {} 
\sphinxstyleliteralstrong{\sphinxupquote{beta}} (\sphinxstyleliteralemphasis{\sphinxupquote{float}}) \textendash{} 1/kbT, the default is at T=298K

\item {} 
\sphinxstyleliteralstrong{\sphinxupquote{uji}} (\sphinxstyleliteralemphasis{\sphinxupquote{np.ndarray}}) \textendash{} The energy matrix, is zero for unbiased simulation (S,Ntot)

\item {} 
\sphinxstyleliteralstrong{\sphinxupquote{fi0}} (\sphinxstyleliteralemphasis{\sphinxupquote{np.ndarray}}) \textendash{} The initial guess of fi (\sphinxhyphen{}ln(Zi/Z0)) for optimization (S,)

\end{itemize}

\end{description}\end{quote}
\index{initialize() (wham.Uwham.Uwham method)@\spxentry{initialize()}\spxextra{wham.Uwham.Uwham method}}

\begin{fulllineitems}
\phantomsection\label{\detokenize{wham:wham.Uwham.Uwham.initialize}}\pysiglinewithargsret{\sphinxbfcode{\sphinxupquote{initialize}}}{}{}
initialize some parameters of the class
:returns:
\begin{enumerate}
\sphinxsetlistlabels{\arabic}{enumi}{enumii}{}{.}%
\item {} 
uji= beta*k*0.5*(n\sphinxhyphen{}nstar)**2 (S,Ntot)

\item {} 
fi0=initial guesses for Uwham (S,)

\end{enumerate}

\end{fulllineitems}

\index{self\_consistent() (wham.Uwham.Uwham method)@\spxentry{self\_consistent()}\spxextra{wham.Uwham.Uwham method}}

\begin{fulllineitems}
\phantomsection\label{\detokenize{wham:wham.Uwham.Uwham.self_consistent}}\pysiglinewithargsret{\sphinxbfcode{\sphinxupquote{self\_consistent}}}{\emph{maxiter=100000.0}, \emph{tol=1e\sphinxhyphen{}08}, \emph{print\_every=\sphinxhyphen{}1}}{}
performs self\sphinxhyphen{}consistent iterations of unbinned Wham
:param maxiter: specifies the maximum number of self consistent iterations are allowed
:type maxiter: int
:param tol: specifies the tolerance of the iteration
:type tol: float
:param print\_every: The frequency at which the programs outputs the result. If the number is less than zero, the program will not output result.
:type print\_every: int
\begin{description}
\item[{Returns}] \leavevmode\begin{enumerate}
\sphinxsetlistlabels{\arabic}{enumi}{enumii}{}{.}%
\item {} 
wji=the weights of all the observations in the simulation (Ntot,)

\item {} 
fi=\sphinxhyphen{}ln(Zi/Z0) (S,)

\end{enumerate}

\end{description}

\end{fulllineitems}

\index{Maximum\_likelihood() (wham.Uwham.Uwham method)@\spxentry{Maximum\_likelihood()}\spxextra{wham.Uwham.Uwham method}}

\begin{fulllineitems}
\phantomsection\label{\detokenize{wham:wham.Uwham.Uwham.Maximum_likelihood}}\pysiglinewithargsret{\sphinxbfcode{\sphinxupquote{Maximum\_likelihood}}}{\emph{ftol=2.22e\sphinxhyphen{}09}, \emph{gtol=1e\sphinxhyphen{}05}, \emph{maxiter=15000}, \emph{maxfun=15000}, \emph{disp=None}, \emph{iprint=\sphinxhyphen{}1}}{}
Optimizes the negative likelihood equation using LBFGS algorithm
\begin{quote}\begin{description}
\item[{Parameters}] \leavevmode\begin{itemize}
\item {} 
\sphinxstyleliteralstrong{\sphinxupquote{ftol}} (\sphinxstyleliteralemphasis{\sphinxupquote{float}}) \textendash{} the tolerance as set forth by scipy.minimize (default 2.22e\sphinxhyphen{}09)

\item {} 
\sphinxstyleliteralstrong{\sphinxupquote{gtol}} (\sphinxstyleliteralemphasis{\sphinxupquote{float}}) \textendash{} the tolerance as set forth by scipy.minimize (default 1e\sphinxhyphen{}05)

\item {} 
\sphinxstyleliteralstrong{\sphinxupquote{maxiter}} (\sphinxstyleliteralemphasis{\sphinxupquote{int}}) \textendash{} the maximum number of iterations as set forth by scipy.minimize. (default 15000)

\item {} 
\sphinxstyleliteralstrong{\sphinxupquote{maxfun}} (\sphinxstyleliteralemphasis{\sphinxupquote{int}}) \textendash{} the maximum number of function evaluations as set forth by scipy.minimize. (default 15000)

\item {} 
\sphinxstyleliteralstrong{\sphinxupquote{iprint}} (\sphinxstyleliteralemphasis{\sphinxupquote{int}}) \textendash{} the frequency at which the program outputs the result. Will not output if less than 0. (default \sphinxhyphen{}1)

\end{itemize}

\item[{Returns}] \leavevmode
the optimal weights for each observation if converged else None

\item[{Return type}] \leavevmode
wji(np.ndarray)

\end{description}\end{quote}

\end{fulllineitems}

\index{compute\_betaF\_profile() (wham.Uwham.Uwham method)@\spxentry{compute\_betaF\_profile()}\spxextra{wham.Uwham.Uwham method}}

\begin{fulllineitems}
\phantomsection\label{\detokenize{wham:wham.Uwham.Uwham.compute_betaF_profile}}\pysiglinewithargsret{\sphinxbfcode{\sphinxupquote{compute\_betaF\_profile}}}{\emph{min}, \emph{max}, \emph{bins=100}}{}
Function that calculates the Free energy for Uwham from the observations xji and
weights wji
\begin{quote}\begin{description}
\item[{Parameters}] \leavevmode\begin{itemize}
\item {} 
\sphinxstyleliteralstrong{\sphinxupquote{min}} (\sphinxstyleliteralemphasis{\sphinxupquote{float}}) \textendash{} the minimum of the binned vector (float/int)

\item {} 
\sphinxstyleliteralstrong{\sphinxupquote{max}} (\sphinxstyleliteralemphasis{\sphinxupquote{float}}) \textendash{} the maximum of the binned vector (float/int)

\item {} 
\sphinxstyleliteralstrong{\sphinxupquote{bins}} (\sphinxstyleliteralemphasis{\sphinxupquote{int}}) \textendash{} number of bins

\end{itemize}

\item[{Returns}] \leavevmode
\begin{enumerate}
\sphinxsetlistlabels{\arabic}{enumi}{enumii}{}{.}%
\item {} 
bins\_vec= binned vector from min to max (bins\sphinxhyphen{}1,)

\item {} 
p=the probability in each bin

\item {} 
F=The free energy in the binned vectors from min to max for all the simulations performed(S,bins\sphinxhyphen{}1)

\end{enumerate}


\end{description}\end{quote}

\end{fulllineitems}

\index{get\_pji() (wham.Uwham.Uwham method)@\spxentry{get\_pji()}\spxextra{wham.Uwham.Uwham method}}

\begin{fulllineitems}
\phantomsection\label{\detokenize{wham:wham.Uwham.Uwham.get_pji}}\pysiglinewithargsret{\sphinxbfcode{\sphinxupquote{get\_pji}}}{}{}~\begin{description}
\item[{Function that obtains all the weights for unbiased as well as biased simulations following the equation}] \leavevmode
pji\_k = np.exp(fi)*np.exp(\sphinxhyphen{}Uji\_k)*wji

\end{description}

where wji is the unbiased weights
\begin{quote}\begin{description}
\item[{Returns}] \leavevmode
pji matrix with shape (S,Ntot)

\item[{Return type}] \leavevmode
pji(np.ndarray)

\end{description}\end{quote}

\end{fulllineitems}


\end{fulllineitems}

\index{Uwham\_NLL\_eq() (in module wham.Uwham)@\spxentry{Uwham\_NLL\_eq()}\spxextra{in module wham.Uwham}}

\begin{fulllineitems}
\phantomsection\label{\detokenize{wham:wham.Uwham.Uwham_NLL_eq}}\pysiglinewithargsret{\sphinxcode{\sphinxupquote{wham.Uwham.}}\sphinxbfcode{\sphinxupquote{Uwham\_NLL\_eq}}}{\emph{fi}, \emph{uji}, \emph{Ni}}{}~\begin{quote}\begin{description}
\item[{Parameters}] \leavevmode\begin{itemize}
\item {} 
\sphinxstyleliteralstrong{\sphinxupquote{fi}} (\sphinxstyleliteralemphasis{\sphinxupquote{np.ndarray}}) \textendash{} initial guess of the log of the partition coefficients Zk normalized by Z0 e.g. f1 = \sphinxhyphen{}ln(Z1/Z0) (shape (S,))

\item {} 
\sphinxstyleliteralstrong{\sphinxupquote{uji}} (\sphinxstyleliteralemphasis{\sphinxupquote{np.ndarray}}) \textendash{} beta*Wji energy matrix (shape(S,Ntot))

\item {} 
\sphinxstyleliteralstrong{\sphinxupquote{Ni}} (\sphinxstyleliteralemphasis{\sphinxupquote{np.ndarray}}) \textendash{} the count of observations in each simulation (shape(S,))

\end{itemize}

\item[{Returns}] \leavevmode
Negative Log Likelihood value of Uwham

\end{description}\end{quote}

\end{fulllineitems}



\renewcommand{\indexname}{Python Module Index}
\begin{sphinxtheindex}
\let\bigletter\sphinxstyleindexlettergroup
\bigletter{w}
\item\relax\sphinxstyleindexentry{wham.Bwham}\sphinxstyleindexpageref{wham:\detokenize{module-wham.Bwham}}
\item\relax\sphinxstyleindexentry{wham.lib.numeric}\sphinxstyleindexpageref{wham.lib:\detokenize{module-wham.lib.numeric}}
\item\relax\sphinxstyleindexentry{wham.lib.wham\_utils}\sphinxstyleindexpageref{wham.lib:\detokenize{module-wham.lib.wham_utils}}
\item\relax\sphinxstyleindexentry{wham.Uwham}\sphinxstyleindexpageref{wham:\detokenize{module-wham.Uwham}}
\end{sphinxtheindex}

\renewcommand{\indexname}{Index}
\printindex
\end{document}